\begin{hcarentry}[updated]{Haskell in Gentoo Linux}
\label{gentoo}
\report{Lennart Kolmodin}%11/09
\makeheader

Gentoo Linux currently officially supports GHC 6.10.4, including the
latest Haskell Platform~\cref{haskell-platform} for x86, amd64, sparc
and ppc64.  For previous GHC versions we also have binaries available
for alpha, hppa and ia64.

The full list of packages available through the official repository
can be viewed at
\url{http://packages.gentoo.org/category/dev-haskell?full\_cat}.

The GHC architecture/version matrix is available at
\url{http://packages.gentoo.org/package/dev-lang/ghc}.

Please report problems in the normal Gentoo bug tracker
at \url{bugs.gentoo.org}.

We have also recently started an official Gentoo Haskell blog where we
can communicate with our users what we are doing
\url{http://gentoohaskell.wordpress.com/}.

There is also an overlay which contains more than 300 extra unofficial
and testing packages. Thanks to the Haskell developers using Cabal and
Hackage~\cref{hackagedb}, we have been able to write a tool called
``hackport'' (initiated by Henning G\"unther) to generate Gentoo
packages with minimal user intervention.  Notable packages in the
overlay include the latest version of the Haskell Platform as well as
the latest 6.12.2 release of GHC, as well as popular Haskell packages
such as pandoc and gitit.

More information can be found about the Gentoo Haskell Overlay at
\url{http://haskell.org/haskellwiki/Gentoo}. Using Darcs~\cref{darcs},
it is easy to keep updated and submit new packages and to fix any
problems in existing packages.  It is also available via the Gentoo
overlay manager ``layman''.  If you choose to use the overlay, then
any problems should be reported on IRC (\verb+#gentoo-haskell+ on
freenode), where we coordinate development, or via email
\email{haskell@gentoo.org} (as we have more people with the ability to
fix the overlay packages that are contactable in the IRC channel than
via the bug tracker).

Through recent efforts we have devoped a tool called
``haskell-updater''
\url{http://www.haskell.org/haskellwiki/Gentoo#haskell-updater}
(initiated by Ivan Lazar Miljenovic).  This is a replacement of the
old \verb+ghc-updater+ script for rebuilding packages when a new
version of GHC is installed which is now not only written in Haskell
but will also rebuild broken packages.  ``haskell-updater'' is still
in active development to further refine and add to its features and
capabilities.

As always we are more than happy for (and in fact encourage) Gentoo
users to get involved and help us maintain our tools and packages,
even if it's as simple as reporting packages that do not always work
or need updating: with such a wide range of GHC and package versions
to co-ordinate, it's hard to keep up!  Please contact us on IRC or
email if you are interested!
\end{hcarentry}
