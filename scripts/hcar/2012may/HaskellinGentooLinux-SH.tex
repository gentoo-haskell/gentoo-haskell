% HaskellinGentooLinux-SH.tex
\begin{hcarentry}[updated]{Haskell in Gentoo Linux}
\label{gentoo}
\report{Sergei Trofimovich}%05/12
\makeheader

Gentoo Linux currently officially supports GHC 7.4.1, GHC 7.0.4 and GHC 6.12.3
on x86, amd64, sparc, alpha, ppc, ppc64 and some arm platforms.

The full list of packages available through the official repository
can be viewed at
\url{http://packages.gentoo.org/category/dev-haskell?full\_cat}.

The GHC architecture/version matrix is available at
\url{http://packages.gentoo.org/package/dev-lang/ghc}.

Please report problems in the normal Gentoo bug tracker
at \url{bugs.gentoo.org}.

There is also an overlay which contains almost 800 extra unofficial
and testing packages. Thanks to the Haskell developers using Cabal and
Hackage~\cref{hackagedb}, we have been able to write a tool called
``hackport'' (initiated by Henning G\"unther) to generate Gentoo
packages with minimal user intervention. Notable packages in the
overlay include the latest version of the Haskell Platform \cref{haskell-platform} as well as
the latest 7.4.1 release of GHC, as well as popular Haskell packages
such as pandoc~\cref{pandoc}, gitit~\cref{gitit}, yesod~\cref{yesod} and others.

As usual GHC 7.4 branch required some packages to be patched. For a 6
months period we have got about 150 patches waiting for upstream inclusion.

Over the time more and more people get involved in gentoo-haskell project
which reflects positively on haskell ecosystem health status.

More information about the Gentoo Haskell Overlay can be found at
\url{http://haskell.org/haskellwiki/Gentoo}. It is available via the Gentoo
overlay manager ``layman''.  If you choose to use the overlay, then
any problems should be reported on IRC (\verb+#gentoo-haskell+ on
freenode), where we coordinate development, or via email
\email{haskell@@gentoo.org} (as we have more people with the ability to
fix the overlay packages that are contactable in the IRC channel than
via the bug tracker).

As always we are more than happy for (and in fact encourage) Gentoo
users to get involved and help us maintain our tools and packages,
even if it is as simple as reporting packages that do not always work
or need updating: with such a wide range of GHC and package versions
to co-ordinate, it is hard to keep up!  Please contact us on IRC or
email if you are interested!

For concrete tasks see our perpetual TODO list:
\url{https://github.com/gentoo-haskell/gentoo-haskell/blob/master/projects/doc/TODO.rst}
\end{hcarentry}
