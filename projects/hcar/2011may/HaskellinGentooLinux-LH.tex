% HaskellinGentooLinux-LH.tex
\begin{hcarentry}{Haskell in Gentoo Linux}
\label{gentoo}
\report{Sergei Trofimovich}%05/11
\makeheader

Gentoo Linux currently officially supports GHC 6.12.3 on x86, amd64, sparc,
ppc, ppc64, alpha and ia64. Hppa support was dropped.

GHC also runs on gentoo-hardened \url{http://www.gentoo.org/proj/en/hardened/}
and on some gentoo-alt \url{http://www.gentoo.org/proj/en/gentoo-alt/}
systems. They are freebsd, macos-prefix and solaris-prefix for now.
Special thanks to Fabian Groffen and the Prefix Team.

The full list of packages available through the official repository
can be viewed at
\url{http://packages.gentoo.org/category/dev-haskell?full\_cat}.

The GHC architecture/version matrix is available at
\url{http://packages.gentoo.org/package/dev-lang/ghc}.

Please report problems in the normal Gentoo bug tracker
at \url{bugs.gentoo.org}.

There is also an overlay which contains more than 600 extra unofficial
and testing packages. Thanks to the Haskell developers using Cabal and
Hackage~\cref{hackagedb}, we have been able to write a tool called
``hackport'' (initiated by Henning G\"unther) to generate Gentoo
packages with minimal user intervention. Notable packages in the
overlay include the latest version of the Haskell Platform as well as
the latest 7.0.3 release of GHC, as well as popular Haskell packages
such as pandoc~\cref{pandoc} and gitit~\cref{gitit}.

Due to tremendous amount of work done by Mark Wright most of packages
work with GHC 7.0.3.

All Gentoo Haskell projects moved to \url{https://github.com/gentoo-haskell}
where one can find new home of our overlay and tools helping keep overlay
up-to-date.

More information about the Gentoo Haskell Overlay can be found at
\url{http://haskell.org/haskellwiki/Gentoo}. It is available via the Gentoo
overlay manager ``layman''.  If you choose to use the overlay, then
any problems should be reported on IRC (\verb+#gentoo-haskell+ on
freenode), where we coordinate development, or via email
\email{haskell@@gentoo.org} (as we have more people with the ability to
fix the overlay packages that are contactable in the IRC channel than
via the bug tracker).

As always we are more than happy for (and in fact encourage) Gentoo
users to get involved and help us maintain our tools and packages,
even if it is as simple as reporting packages that do not always work
or need updating: with such a wide range of GHC and package versions
to co-ordinate, it is hard to keep up!  Please contact us on IRC or
email if you are interested!
\end{hcarentry}
