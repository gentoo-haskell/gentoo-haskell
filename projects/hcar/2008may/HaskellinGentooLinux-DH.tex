\begin{hcarentry}[updated]{Haskell in Gentoo Linux}
\label{gentoo}
\report{Lennart Kolmodin}
\makeheader

GHC version 6.8.2 has been in Gentoo since late last year, and is about to
go stable. All of the 60+ Haskell libraries and tools work with it too. 
There are also GHC binaries available for alpha, amd64, hppa, ia64, sparc
and x86.

Browse the packages in portage at 
\url{http://packages.gentoo.org/category/dev-haskell?full\_cat}.

The GHC architecture/version matrix is available at
\url{http://packages.gentoo.org/package/dev-lang/ghc}.

Please report problems in the normal Gentoo bug tracker
at \url{bugs.gentoo.org}.

There is also a Haskell overlay providing another 200 packages. Thanks to
the recent progress of Cabal and Hackage, we have written a tool called
``hackport'' (initiated by Henning G\"unther) to generate Gentoo packages,
that rarely need much tweeking.

The overlay is available at
\url{http://haskell.org/haskellwiki/Gentoo}. Using
darcs~\cref{darcs} it's easy to keep updated and send patches.
It's also available via the Gentoo overlay manager ``layman''.
If you choose to use the overlay then problems should be
reported on
IRC (\verb+#gentoo-haskell+ on freenode), where we coordinate
development, or via email \email{haskell@gentoo.org}.

Lately a few of our developers have shifted focus, and only a few
developers remain. If you would like to help, which would include
working on the Gentoo Haskell framework, hacking on hackport, writing
ebuilds and supporting users, please contact us on IRC or email as noted
above.
\end{hcarentry}
