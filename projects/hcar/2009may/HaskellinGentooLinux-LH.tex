\begin{hcarentry}[updated]{Haskell in Gentoo Linux}
\label{gentoo}
\report{Lennart Kolmodin}%05/09
\makeheader

Gentoo Linux is working towards supporting GHC 6.10.3 and the Haskell
Platform~\cref{haskell-platform}, and at the time of writing both are available hard masked in portage.
For previous GHC versions we have binaries available for alpha, amd64, hppa,
ia64, sparc, and x86.

Browse the packages in portage at 
\url{http://packages.gentoo.org/category/dev-haskell?full\_cat}.

The GHC architecture/version matrix is available at
\url{http://packages.gentoo.org/package/dev-lang/ghc}.

Please report problems in the normal Gentoo bug tracker
at \url{bugs.gentoo.org}.

There is also a Haskell overlay providing another 300 packages. Thanks to
the haskell developers using Cabal and Hackage~\cref{hackagedb}, we have been
able to write a tool called ``hackport'' (initiated by Henning G\"unther) to
generate Gentoo packages that rarely need much tweaking.

The overlay is available at
\url{http://haskell.org/haskellwiki/Gentoo}. Using
Darcs~\cref{darcs}, it is easy to keep updated and send patches.
It is also available via the Gentoo overlay manager ``layman''.
If you choose to use the overlay, then problems should be
reported on
IRC (\verb+#gentoo-haskell+ on freenode), where we coordinate
development, or via email \email{haskell@gentoo.org}.

Lately a few of our developers have shifted focus, and only a few
developers remain. If you would like to help, which would include
working on the Gentoo Haskell framework, hacking on hackport, writing
ebuilds, and supporting users, please contact us on IRC or email as noted
above.
\end{hcarentry}
