\begin{hcarentry}[updated]{Haskell in Gentoo Linux}
\label{gentoo}
\report{Duncan Coutts, Andres L\"oh}
\entry{updated}% done, 2007-12-22
\makeheader

GHC version 6.8.2 is now in Gentoo and (almost) all of the 60+ Haskell
libraries and tools work with it too. This includes the latest versions
of all the ``extralibs'' that were released with GHC 6.8.x. There are also
GHC binaries available for amd64, sparc and x86.

The fact that all 60+ packages work is no trivial matter since many of
them do not yet have official releases that work with ghc-6.8. The
packages in Gentoo that do not yet work with 6.8 are hmake, wxhaskell
and hs-plugins. In the case of the last two this is a long standing
issue since they have no releases that work with ghc-6.6.1 either.

The current set of packages in portage is:
{%
\texttt{alex},
\texttt{alut},
\texttt{arrows},
\texttt{binary},
\texttt{bzlib},
\texttt{c2hs},
\texttt{cabal},
\texttt{cgi},
\texttt{cpphs},
\texttt{darcs},
\texttt{drift},
\texttt{fgl},
\texttt{filepath},
\texttt{frown},
\texttt{glut},
\texttt{gtk2hs},
\texttt{haddock},
\texttt{happy},
\texttt{harp},
\texttt{haskell-src},
\texttt{haskell-src-exts},
\texttt{haxml},
\texttt{hdbc},
\texttt{hdbc-odbc},
\texttt{hdbc-postgresql},
\texttt{hdbc-sqlite},
\texttt{hdoc},
\texttt{hmake},
\texttt{hscolour},
\texttt{hslogger},
\texttt{hs-plugins},
\texttt{hsql},
\texttt{hsql-mysql},
\texttt{hsql-odbc},
\texttt{hsql-postgresql},
\texttt{hsql-sqlite},
\texttt{hsshellscript},
\texttt{html},
\texttt{http},
\texttt{hunit},
\texttt{hxt},
\texttt{iconv},
\texttt{lhs2tex},
\texttt{missingh},
\texttt{mtl},
\texttt{network},
\texttt{openal},
\texttt{opengl},
\texttt{parallel},
\texttt{quickcheck},
\texttt{regex-base},
\texttt{regex-compat},
\texttt{regex-posix},
\texttt{rss},
\texttt{stm},
\texttt{time},
\texttt{uuagc},
\texttt{uulib},
\texttt{wash},
\texttt{wxhaskell},
\texttt{x11},
\texttt{xhtml},
\texttt{xmobar},
\texttt{xmonad},
\texttt{xmonad-contrib},
\texttt{zlib}}.

See also:
\begin{compactitem}
\item \url{http://packages.gentoo.org/category/dev-haskell}
\item \url{http://packages.gentoo.org/package/dev-lang/ghc}
\end{compactitem}

Please report problems in the normal Gentoo bug tracker
at \url{bugs.gentoo.org}.

There are a further 150 packages in the Haskell overlay at
\url{http://haskell.org/haskellwiki/Gentoo}. There you can
access and test the latest versions of the ebuilds, and
send patches using darcs~\cref{darcs}. The overlay is
also available via the Gentoo overlay manager ``layman''.
If you choose to use the overlay then problems should be
reported on
IRC (\verb+#gentoo-haskell+ on freenode), where we coordinate
development, or via email \email{haskell@gentoo.org}.

We are having some difficulty deciding which are the more
popular packages that would be worth adding to portage. If you
have suggestions, please contact us.
\end{hcarentry}
